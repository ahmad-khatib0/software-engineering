\documentclass[12pt]{extarticle}
\usepackage[document]{ragged2e}
\usepackage[utf8]{inputenc}
\usepackage{amsmath}
\usepackage[left=0.5in, right=0.5in, top=1in, bottom=1in]{geometry}
\usepackage[colorlinks=true, linkcolor=blue, urlcolor=red]{hyperref}
\usepackage{times} % Times New Roman

% the following is for the code
\usepackage{listings}
\usepackage{color}
\definecolor{dkgreen}{rgb}{0,0.6,0}
\definecolor{gray}{rgb}{0.5,0.5,0.5}
\definecolor{mauve}{rgb}{0.58,0,0.82}
\lstset{frame=tb,
  language=Java,
  aboveskip=3mm,
  belowskip=3mm,
  showstringspaces=false,
  columns=flexible,
  basicstyle={\small\ttfamily},
  numbers=none,
  numberstyle=\tiny\color{gray},
  keywordstyle=\color{blue},
  commentstyle=\color{dkgreen},
  stringstyle=\color{mauve},
  breaklines=true,
  breakatwhitespace=true,
  tabsize=3
}

\begin{document}
\textbf{Representing Groups of Bits}\\
Even with binary, sometimes we have so many bits that the number is unreadable. In those cases,
we use hexadecimal digits to specify bit patterns. The hexadecimal system has 16 digits, each
of which can represent one group of 4 bits.

\vspace{18pt}
The octal system, based on the number eight, is less common, but you will encounter it occasionally.
The eight octal digits span from 0 to 7, and each one represents a group of three bits. Table 2-2

\vspace{18pt}
\large{Using Hexadecimal Digits}\\
Hexadecimal digits are especially convenient when we need to specify the state of a group of,
say, 16 or 32 switches. In place of each group of four bits, we can write one hexadecimal digit.
Here are two examples:
\center 6c2a = 0110110000101010, 0123abcd = 00000001001000111010101111001101

\vspace{18pt}
A single bit isn’t usually useful for storing data. The smallest number of bits that can be
accessed at a time in a computer is defined as a byte. In most modern computers, a byte
consists of eight bits, but there are exceptions to the eight-bit byte. For example, the
CDC 6000 series of scientific mainframe computers used a six-bit byte.

\vspace{18pt}
In the C and C++ programming languages, prefixing a number with \textbf{0x} that’s a zero and 
a lowercase x — specifies that the number is expressed in hexadecimal, and prefixing a number 
with only a 0 specifies octal. C++ allows us to specify a value in binary by prefixing the 
number with 0b. Although the 0b notation for specifying binary is not part of standard C,
our compiler, gcc, allows it. Thus, when writing C or C++ code in this book, these all mean
the same thing: \\
\Large\textbf 100 = 0x64 = 0144 = 0b01100100

\vspace{18pt}
In the decimal number system, the integer 123 is taken to mean: \\
$1 \times 100 + 2 \times 10 + 3 $  \hspace{4pt} or: $1 \times 10^2 + 2 \times 10^1 + 3 \times 10^0 $\\
the rightmost digit, 3, is the least significant digit because its value contributes the least
to the number’s total value. The leftmost digit, 1, is the most significant digit because it
contributes the most value.

\vspace{18pt}
The radix in the binary number system is 2, so there are only two symbols for representing
the digits;  this means that di = 0, 1, and we can write this expression as follows:
$ d_{n-1} * 2^{n-1} + d_{n–2} * 2^{n–2} + ... + d_1 * 2^1 + d_0 * 2^0 $

\vspace{18pt}
\textbf{Converting Binary to Unsigned Decimal}
You can easily convert from binary to decimal by computing the value of 2 raised to the power
of  the position it is in and then multiplying that by the value of the bit in that position.
$ 10010101_2 = 1 * 2^7 + 0 * 2^6 + 0 * 2^5 + 1 * 2^4 + 0 × 2^3 + 1 * 2^2 + 0 * 2^1 + 1 * 2^0 $\\
$ = 128 + 16 + 4 + 1  \hspace{44pt} 149_{10} $

\vspace{18pt}
\textbf{Converting Unsigned Decimal to Binary}
If we want to convert an unsigned decimal integer, N, to binary, we set it equal to
the previous expression for binary numbers to give this equation: \\
$N = d_{n–1} * 2^{n–1} + d_{n–2} * 2^{n–2} + … + d_1 × 2^1 + d_0 × 2^0$ \\
where each $d_i$ = 0 or 1. We divide both sides of this equation by 2, and the
exponent of each 2 term on the right side decreases by 1, giving the following: \\
$N_1 + \frac{r_0}{2} = (d_{n–1} * 2^{n–2} * d_{n–2} * 2^{n–3} + … + d_1 × 2^0) + d_0 * 2^{–1}$ \\
where $N_1$ is the integer part, and the remainder, r 0, is 0 for even numbers and 1 for odd
numbers. Doing a little rewriting, we have the equivalent equation: \\
$ N_1 + \frac{r_0}{2} = (d_{n–1} * 2^{n–2} + d_{n–2} * 2^{n–3} + … + d_1 × 2^0) + \frac{d_0}{2}$

\vspace{8pt}
All the terms within the parentheses on the right side are integers. The integer part of both
sides of an equation must be equal, and the fractional parts must also be equal. That is:\\
$ N_1 = d_{n–1} * 2^{n–2} + d_{n–2} * 2^{n–3} + … + d_1 × 2^0$ \\
and $\frac{r_0}{2} = \frac{d_0}{2}$ \newline \newline
Thus, we see that $ d_0 = r_0 $ . Subtracting $r_0 / 2$ (which equals $d_0 / 2$ ) from
both sides of our expanded equation gives this: \\
$ N_1 = d_{n–1} * 2^{n–2} + d_{n–2} * 2^{n–3} + … + d_1 × 2^0 $ \\
Again, we divide both sides by 2: \newline\newline
$N_2  + \frac{r_1}{2}  = d_{n-1} * 2^{n-3} + d_{n-2} * 2^{n-4} + ... + d_2 * 2^0 + d_1 * 2^{-1}$ \\
= $(d_{n-1}  * 2^{n-3} + d_{n-2} * 2^{n-4} + ... + d_2 * 2^0 ) \frac{d_1}{2}$ \newline\newline
Using the same reasoning as earlier, $d_1 = r^1$. We can produce the binary
representation of a number by working from right to left, repeatedly dividing by 2, and using
the remainder as the value of the respective bit. This is summarized in the following algorithm: \\

\begin{lstlisting}
quotient = N
i = 0
d_1 = quotient % 2
quotient = quotient % 2
While quotient != 0
    i = i + 1
    d_i = quotient % 2
    quotient = quotient / 2
\end{lstlisting}

The name random access memory is misleading. Here random access means that it takes the
same amount of time to access any byte in the memory, not that any randomness is involved 
when reading the byte. We contrast RAM with sequential access memory (SAM), where the amount
of time it takes to access a byte depends on its position in some sequence.

\vspace{10pt}
\Large{Read-only memory (ROM)}\\
ROM is also called nonvolatile memory (NVM). The control unit can read
the state of each bit but can’t change it.

\vspace{10pt}
Computer scientists typically express the address of each byte in memory in hexadecimal, starting
the numbering at zero. Thus, we would say that the 957th byte is at address 0x3bc (= 956 in decimal).

\vspace{10pt}
\begin{itemize}\raggedright
  \item An ASCII character in 8-bit ASCII encoding is 8 bits (1 byte), though it can fit
        in 7 bits.
  \item An ISO-8895-1 character in ISO-8859-1 encoding is 8 bits (1 byte).
  \item A Unicode character in UTF-8 encoding is between 8 bits (1 byte) and 32 bits (4 bytes).
  \item A Unicode character in UTF-16 encoding is between 16 (2 bytes) and 32 bits (4 bytes),
        though most of the common characters take 16 bits. This is the encoding used by Windows internally.
  \item A Unicode character in UTF-32 encoding is always 32 bits (4 bytes).
  \item An ASCII character in UTF-8 is 8 bits (1 byte), and in UTF-16 - 16 bits.
  \item The additional (non-ASCII) characters in ISO-8895-1 (0xA0-0xFF) would take 16 bits
        in UTF-8 and UTF-16.
\end{itemize}

\vspace{18pt}
UTF-8 is backward compatible with an older standard, the American Standard Code for Information
Interchange (ASCII—pronounced “ask-ee”). ASCII uses seven bits to specify each code point 
in a 128-character set, which contains the English alphabet (uppercase and lowercase), 
numerals, special characters, and control characters.

\vspace{18pt}
Hello, World!\textbackslash n , the compiler stores this text string as a constant array 
of characters. To specify the extent of this array, a C-style string uses the code point 
U+0000 (ASCII NUL) at the end of the string as a sentinel value, a unique value that 
indicates the end of a sequence of characters. Thus, the compiler must allocate 13 bytes 
for this string: 11 for Hello, World!, 1 for the newline \textbackslash n and 1 for the NUL.

\vspace{18pt}
C uses U+000A (ASCII LF) as a single newline character (at address 0x4004ae in Table 2-6) 
even though the C syntax requires that the programmer write two characters, \textbackslash n.
The text string ends with the NUL character at 0x4004af. \\
In Pascal, the length of the string is specified by the first byte in the string, which 
is taken to be an eight-bit unsigned integer. (This is the reason for the 256-character 
limit on text strings in Pascal.) The C++ string class has additional features, but the 
actual text string is stored as a C-style text string within the C++ string instance.

\vspace{18pt}
Using only one byte restricts the range of unsigned integers to be from 0 to 255 10, since
ff16 = 255 10. The default size for an unsigned integer in our programming environment is four
bytes, which allows for a range of 0 to \textbf{4,294,967,295}.

\vspace{18pt}
Binary Coded Decimal (BCD) is another code for storing integers. It uses
four bits for each decimal digit, as shown in Table 2-7.

\vspace{18pt}
With only 10 of the possible 16 combinations being used (in BCD), we can see that six bit
patterns are wasted. This means that a 16-bit storage location has a range of 0 to 9,999 
for values if we use BCD, compared to a range of 0 to 65,535 if we use binary, so this is 
a less efficient use of memory. On the other hand, the conversions between a character 
format and an integer format are simpler with BCD,

\vspace{18pt}
BCD is important in specialized systems that deal primarily with numerical business data,
because they tend to print numbers more often than perform mathematical operations on them.
COBOL, a programming language intended for business applications, supports a packed BCD format
where two digits (in BCD code) are stored in each eight-bit byte. Here, the last (four-bit)
digit is used to store the sign of the number,

\vspace{18pt}
A header file is used to provide a prototype statement for each function, which specifies
these data types. The stdio.h header file defines the interface to many of the functions 
in the C standard library, which allows the compiler to know what to do when calls to any 
of these functions are encountered in our source code.

\vspace{18pt}
\textbf{Little-Endian}\\
Data is stored in memory with the least significant byte in a multiple-byte
value in the lowest-numbered address. That is, the “littlest” byte (counts the
least) comes first in memory.
When we examine memory one byte at a time, each byte is displayed in
numerically ascending addresses

\vspace{18pt}
\textbf{Big-Endian}\\
Data is stored in memory with the most significant byte in a multiple-byte
value in the lowest-numbered address. That is, the “biggest” byte (counts
the most) comes first in memory.

\vspace{18pt}
\textbf{Endianness} is not an issue. However, you need to know about it because it can 
be confusing when examining memory in the debugger. Endianness is also an issue when 
different computers are communicating with each other. For example, Transport Control 
Protocol/Internet Protocol (TCP/IP) is defined to be big-endian, sometimes called network
byte order. The x86-64 architecture is little-endian. The operating system reorders the 
bytes for internet communication. But if you’re writing communications software for an 
operating system itself or for an embedded system that may not have an operating system, 
you need to know about byte order.

\vspace{18pt}
\begin{itemize}
\item Bits  A computer is a collection of on/off switches that we can represent with bits.
\item Hexadecimal  A number system based on 16. Each hexadecimal digit, 0 to f, 
represents four bits.
\item Byte  A group of eight bits. The bit pattern can be expressed as two hexadecimal digits.
\item Converting between decimal and binary   The two number systems are 
mathematically equivalent.
\item Memory addressing   Bytes in memory are numbered (addressed) sequentially. 
  The byte’s address is usually expressed in hexadecimal.
\item Endianness  An integer that is more than one byte can be stored with the 
  highest-order byte in the lowest byte address (big-endian) or with the lowest-order byte
  in the lowest byte address (little-endian). The x86-64 architecture is little-endian.
\item UTF-8 encoding   A code for storing characters in memory.
\item String  This C-style string is an array of characters terminated by the NUL character.
\item printf  This C library function is used to write formatted data on the monitor screen.
\item scanf  This C library function is used to read formatted data from the keyboard.
\end{itemize}

\end{document}
