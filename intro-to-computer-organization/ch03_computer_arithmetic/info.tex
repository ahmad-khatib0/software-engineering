\documentclass[12pt]{extarticle}
\usepackage{mypreamble}
\usepackage{clrscode4e}

\begin{document}
The carry flag (CF) and overflow flag (OF) in the status flags portion of the CPU’s
rflags register allow us to detect when adding and subtracting binary numbers yields
results that exceed the allocated number of bits for the data type.

\textbf{Subtracting in the Decimal Number System}
\begin{lstlisting}
let borrow = 0
Repeat for i = 0,···,(N − 1)
   If y_i ≤ x_i
      Let difference_i = x_i − y_i
   Else
      Let j = i + 1
      While(x_j  = 0) and (j < N)
        Add 1 to j
      If j = N
        Let borrow = 1
        Subtract 1 from j
        Add 10 to x_j
      While j > i 
        Subtract 1 from x_j
        Subtract 1 from j
        Add 10 to x_j
          Let difference_i = x_i – y_i
\end{lstlisting}

\textbf{YOUR TURN}
\begin{itemize}
  \item How many bits are required to store a single decimal digit? Invent a code for 
    storing eight decimal digits in 32 bits. Using this code, does binary addition produce
    the correct results? You’ll see such a code later in the book and some reasons for 
    its usefulness.
  \item Develop an algorithm for adding fixed-width integers in the binary number system.
  \item Develop an algorithm for adding fixed-width integers in the hexadecimal number system.
  \item Develop an algorithm for subtracting fixed-width integers in the binary number system.
  \item Develop an algorithm for subtracting fixed-width integers in the hexadecimal number system.
\end{itemize}

\vspace{18pt}
\textbf{Two’s Complement}\\
In mathematics, the complement of a quantity is the amount that must be added to make it 
“whole.” When applying this concept to numbers, the definition of whole depends on the 
radix (or base) you’re working in and the number of digits that you allow to represent 
the numbers. If x is an n-digit number in radix r, its radix complement, ¬x, is defined
such that $x + \lnot x = radix^n$, where $radix^n$ is 1 followed by n 0s. For example, if we’re
working with two-digit decimal numbers, then the radix complement of 37 is 63, because \\
$37 + 63 = 102 = 100$. Another way of saying this is that adding a number to its radix 
complement results in 0 with a carry beyond the n digits.\\

\vspace{18pt}
\textbf{Diminished radix complement}\\
Another useful concept is the diminished radix complement, which is defined such that\\
$x + diminished\_radix\_complement = radix^n – 1$. For example, the diminished radix 
complement of 37 is 62, because $37 + 62 = 10^2 – 1 = 99$. If you add a number to 
its diminished radix complement, the result is n of the largest digits in the radix—two
9s in this example of two digits in radix 10.

\begin{tabular}{c|c|c}
\hline 
One hexadecimal digit & Four binary digits (bits) & Signed decimal \\
\hline 
8 &  1000 &  −8 \\ 
9 &  1001 &  −7 \\
a &  1010 &  −6 \\
b &  1011 &  −5 \\
c &  1100 &  −4 \\
d &  1101 &  −3 \\
e &  1110 &  −2 \\
f &  1111 &  −1 \\
0 &  0000 &  0  \\
1 &  0001 &  +1 \\
2 &  0010 &  +2 \\
3 &  0011 &  +3 \\
4 &  0100 &  +4 \\
5 &  0101 &  +5 \\
6 &  0110 &  +6 \\
7 &  0111 &  +7 \\
\hline 
\end{tabular}
% One hexadecimal digit    Four binary digits (bits)     Signed decimal
% 8                               1000                        −8
% 9                               1001                        −7
% a                               1010                        −6
% b                               1011                        −5
% c                               1100                        −4
% d                               1101                        −3
% e                               1110                        −2
% f                               1111                        −1
% 0                               0000                        0
% 1                               0001                        +1
% 2                               0010                        +2
% 3                               0011                        +3
% 4                               0100                        +4
% 5                               0101                        +5
% 6                               0110                        +6
% 7                               0111                        +7

\vspace{18pt}
\begin{itemize}
  \item The high-order bit of each positive number is 0, and the high-order bit
   of each negative number is 1.
  \item Although changing the sign of (negating) a number is more complicated than simply 
   changing the high-order bit, it is common to call the high-order bit the sign bit.
  \item The notation allows for one more negative number than positive numbers.
  \item The range of integers, x, that can be represented in this notation (with four bits)
    is\\ $$ –8_10 \leq x \leq +7_10 $$   \\ 
    or    $$–2^{(4–1)} \leq x \leq + (2^{(4–1)}  – 1)$$ \\ 
    The last observation can be generalized for n bits to the following: 
    $$ –2^{(n–1)} \leq x \leq +(2^{(n–1)} – 1) $$ \\
   When using two’s complement notation, the negative of any n-bit integer, x, 
   is defined as  \\ $$x + (–x) = 2^n$$
\end{itemize}

\end{document}

